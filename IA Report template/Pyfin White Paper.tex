\documentclass[a4paper, justified]{tufte-handout}

\hypersetup{colorlinks}% uncomment this line if you prefer colored hyperlinks (e.g., for onscreen viewing)

%%
% Book metadata
\title{StrataGEM}
\author[]{Invoke Analytics}

%%
% If they're installed, use Bergamo and Chantilly from www.fontsite.com.
% They're clones of Bembo and Gill Sans, respectively.
\IfFileExists{bergamo.sty}{\usepackage[osf]{bergamo}}{}% Bembo
\IfFileExists{chantill.sty}{\usepackage{chantill}}{}% Gill Sans

\usepackage{microtype}
\usepackage{lipsum}
\usepackage{hyperref}

%%
% For nicely typeset tabular material
\usepackage{booktabs}

%%
% For graphics / images
\usepackage{graphicx}
\setkeys{Gin}{width=\linewidth,totalheight=\textheight,keepaspectratio}
\graphicspath{{graphics/}}

% The fancyvrb package lets us customize the formatting of verbatim
% environments.  We use a slightly smaller font.
\usepackage{fancyvrb}
\fvset{fontsize=\normalsize}

%%
% Prints argument within hanging parentheses (i.e., parentheses that take
% up no horizontal space).  Useful in tabular environments.
\newcommand{\hangp}[1]{\makebox[0pt][r]{(}#1\makebox[0pt][l]{)}}

%%
% Prints an asterisk that takes up no horizontal space.
% Useful in tabular environments.
\newcommand{\hangstar}{\makebox[0pt][l]{*}}

%%
% Prints a trailing space in a smart way.
\usepackage{xspace}

% Prints the month name (e.g., January) and the year (e.g., 2008)
\newcommand{\monthyear}{%
  \ifcase\month\or January\or February\or March\or April\or May\or June\or
  July\or August\or September\or October\or November\or
  December\fi\space\number\year
}

% Inserts a blank page
\newcommand{\blankpage}{\newpage\hbox{}\thispagestyle{empty}\newpage}

\usepackage{units}

% Typesets the font size, leading, and measure in the form of 10/12x26 pc.
\newcommand{\measure}[3]{#1/#2$\times$\unit[#3]{pc}}

% Macros for typesetting the documentation
\newcommand{\hlred}[1]{\textcolor{Maroon}{#1}}% prints in red
\newcommand{\hangleft}[1]{\makebox[0pt][r]{#1}}
\newcommand{\hairsp}{\hspace{1pt}}% hair space
\newcommand{\hquad}{\hskip0.5em\relax}% half quad space
\newcommand{\TODO}{\textcolor{red}{\bf TODO!}\xspace}
\newcommand{\na}{\quad--}% used in tables for N/A cells
\providecommand{\XeLaTeX}{X\lower.5ex\hbox{\kern-0.15em\reflectbox{E}}\kern-0.1em\LaTeX}
\newcommand{\tXeLaTeX}{\XeLaTeX\index{XeLaTeX@\protect\XeLaTeX}}
% \index{\texttt{\textbackslash xyz}@\hangleft{\texttt{\textbackslash}}\texttt{xyz}}
\newcommand{\tuftebs}{\symbol{'134}}% a backslash in tt type in OT1/T1
\newcommand{\doccmdnoindex}[2][]{\texttt{\tuftebs#2}}% command name -- adds backslash automatically (and doesn't add cmd to the index)
\newcommand{\doccmddef}[2][]{%
  \hlred{\texttt{\tuftebs#2}}\label{cmd:#2}%
  \ifthenelse{\isempty{#1}}%
    {% add the command to the index
      \index{#2 command@\protect\hangleft{\texttt{\tuftebs}}\texttt{#2}}% command name
    }%
    {% add the command and package to the index
      \index{#2 command@\protect\hangleft{\texttt{\tuftebs}}\texttt{#2} (\texttt{#1} package)}% command name
      \index{#1 package@\texttt{#1} package}\index{packages!#1@\texttt{#1}}% package name
    }%
}% command name -- adds backslash automatically
\newcommand{\doccmd}[2][]{%
  \texttt{\tuftebs#2}%
  \ifthenelse{\isempty{#1}}%
    {% add the command to the index
      \index{#2 command@\protect\hangleft{\texttt{\tuftebs}}\texttt{#2}}% command name
    }%
    {% add the command and package to the index
      \index{#2 command@\protect\hangleft{\texttt{\tuftebs}}\texttt{#2} (\texttt{#1} package)}% command name
      \index{#1 package@\texttt{#1} package}\index{packages!#1@\texttt{#1}}% package name
    }%
}% command name -- adds backslash automatically
\newcommand{\docopt}[1]{\ensuremath{\langle}\textrm{\textit{#1}}\ensuremath{\rangle}}% optional command argument
\newcommand{\docarg}[1]{\textrm{\textit{#1}}}% (required) command argument
\newenvironment{docspec}{\begin{quotation}\ttfamily\parskip0pt\parindent0pt\ignorespaces}{\end{quotation}}% command specification environment
\newcommand{\docenv}[1]{\texttt{#1}\index{#1 environment@\texttt{#1} environment}\index{environments!#1@\texttt{#1}}}% environment name
\newcommand{\docenvdef}[1]{\hlred{\texttt{#1}}\label{env:#1}\index{#1 environment@\texttt{#1} environment}\index{environments!#1@\texttt{#1}}}% environment name
\newcommand{\docpkg}[1]{\texttt{#1}\index{#1 package@\texttt{#1} package}\index{packages!#1@\texttt{#1}}}% package name
\newcommand{\doccls}[1]{\texttt{#1}}% document class name
\newcommand{\docclsopt}[1]{\texttt{#1}\index{#1 class option@\texttt{#1} class option}\index{class options!#1@\texttt{#1}}}% document class option name
\newcommand{\docclsoptdef}[1]{\hlred{\texttt{#1}}\label{clsopt:#1}\index{#1 class option@\texttt{#1} class option}\index{class options!#1@\texttt{#1}}}% document class option name defined
\newcommand{\docmsg}[2]{\bigskip\begin{fullwidth}\noindent\ttfamily#1\end{fullwidth}\medskip\par\noindent#2}
\newcommand{\docfilehook}[2]{\texttt{#1}\index{file hooks!#2}\index{#1@\texttt{#1}}}
\newcommand{\doccounter}[1]{\texttt{#1}\index{#1 counter@\texttt{#1} counter}}

% Generates the index
\usepackage{makeidx}
\makeindex

\pagestyle{fancy}

\begin{document}

% Front matter


% r.1 blank page
%\blankpage

% r.3 full title page
\maketitle


% v.4 copyright page
\begin{fullwidth}

\setlength{\parindent}{0pt}

%\par\smallcaps{Published by \thanklesspublisher}


\end{fullwidth}

% r.5 contents

%\listoffigures

%\listoftables

% r.7 dedication

% r.9 introduction
%\cleardoublepage



\section{Abstract} \label{ch:intro}
StrataGEM is a tool for calculating the optimal allocation of a person's retirement savings in the South African tax environment. It helps an individual to decide on how future savings should be split between different savings options. It considers how much money is already in those accounts, how they grow, income tax, capital gains tax, as well as exemptions for vehicles such as Tax Free Savings Accounts and Retirement Annuities, both during the working and retirement phases of an individual's life. It uses an optimization function to find the optimal allocation of one's savable income to maximise a person's mean income after tax during retirement. The output of the tool is a report telling you how much money you should allocate to each of your savings for the rest of this tax year. It also provides the projects into the future.

\section{The Short Version}
\subsection{Introduction}
\paragraph{Background} This tool was inspired by a Fat Wallet Show podcast on \href{https://justonelap.com/podcast-can-one-etf-rule/}{JustOneLap}, which alerted me to the fact that investing optimally for my retirement is a really difficult problem. Although financial advisors can show that e.g. investing the full 27.5\% allowance in an RA is not the most effective way to structure your savings\sidenote{See for example \href{https://www.fin24.com/Finweek/Investment/save-thousands-in-tax-by-diversifying-your-retirement-savings-20180110}{"Save thousands in tax by diversifying your retirement savings"}, By Danie Venter, published 10 January 2018 on Fin24}, it is very difficult to find the optimal allocation, given the tax implications now and at retirement. For example, the more I put in my RA now, the lower my taxable income, and the more I can save on the same salary. But it does mean that during retirement I'll be paying more income tax than I would if I had put the (admittedly less) money in DIs\sidenote{See \href{http://www.stealthywealth.co.za/2018/05/i-love-capital-gains-tax.html}{this} Stealthy Wealth article on CGT vs income tax, and why you rather want to pay CGT. But there's also the CGT-inflation problem, as discussed later in this paper under \ref{tax rules}{Tax Rules}.}
At Invoke Analytics we solve difficult problems like these by using mathematical algorithms, machine learning, and statistics, and decided to see if we could help. 

\paragraph{Points to note} \newthought{Pyfin was written by engineers} to support our personal finance decisions. Although we spoke to tax practitioners and did tests to ensure that we implemented the tax rules correctly, there are no guarantees. Use at your own risk.

\newthought{Saving large amounts of money} in any of these vehicles will leave you with, well, large amounts of money \sidenote{Provided you use a low fee, high growth vehicle such as ETFs via \href{www.easyequities.co.za}{Easy Equities}}. There are better and worse ways to do it, of course, and that's where StrataGEM comes in. But the key is to save a lot. The calculator can only work with the money you're willing to put away. For more on this, I highly recommend the Fat Wallet Show. And besides, living frugally and saving more will probably make you happier\sidenote{(source)}.

\newthought{"Year zero", when your funds run out}, is extremely sensitive to the withdrawal rate. If you withdraw too much or the investments underperform, you could run out of money years before your life expectancy. The 4\% rule says that backtested over the last hundred years, withdrawing 4\% of capital per year from an investment will ensure that you don't reduce your capital\sidenote{E.g., see \href{http://www.stealthywealth.co.za/2016/06/what-is-4-rule-for-retirement.html}{this} Stealthy Wealth article}. The goal of Pyfin is not to preserve capital, but to maximise income. Therefore the withdrawal rates are often above 4\%. Bear that in mind.

\newthought{StrataGEM's objective is to maximise your mean retirement income after tax}, not to minimize the tax you pay. These two numbers are related, but not identical. The algorithm tries to structure your investments in such a way that your money works for you in the most effective way possible. However, the solutions may be counter-intuitive. It may seem that you pay less tax by sticking to the R33k TFSA limit and using your 27.5\% RA allowance. However, these have tax implications later in life, which may overshadow the tax benefits now (depending on the growth of the investments). 

\newthought{StrataGEM is powered by the PyFin library}, an open-source Python library created for this purpose. It available on Github (XXXX hyperlink to repoXXXX) under the \href{https://opensource.org/licenses/GPL-3.0}{GPL-3.0} the license. The GPL license states that the source code, or any derivative work must be free and must also carry the GPL license, and that the source code must be made available. This does not prevent commercial companies from using StrataGEM, but does make charging for the service or software illegal\sidenote{This does not prevent Invoke Analytics from tailoring StrataGEM to a specific commercial client's needs.}. We call this "copyleft"; the opposite of copyright. Amandla!


\subsection{How it works}
The investment allocation problem is mathematically complex due to the tax rules. Without going into details, it is our strong suspicion that it is mathematically impossible to guarantee that any algorithm will find \textit{the} optimal solution. Deterministic algorithms, which are fast and reliable for some problems, may be reliably underwhelming on these kinds of problems. They tend to get stuck at locally optimal solutions, and never find the globally optimal solution. We therefore employ what are called "metaheuristic" methods\sidenote{For more, see \href{https://en.wikipedia.org/wiki/Metaheuristic}{this} Wikipedia article}. These methods are probabilistic, not deterministic. This means that although you can't guarantee that the algorithm found the best solution, you can be sure that (when tuned properly) it will always find a really good solution, usually leapfrogging the places where deterministic algorithms get stuck. Maybe a solution fond this way is the best; no one could ever know (otherwise, why use an algorithm?). But experience has shown that the solutions these methods find are better than any a human could devise.

There are many heuristic methods, most of which are inspired by natural processes (\href{https://en.wikipedia.org/wiki/Particle_swarm_optimization}{particle swarm}, \href{https://en.wikipedia.org/wiki/Simulated_annealing}{simulated annealing}, etc.) The one we use is called the \href{https://en.wikipedia.org/wiki/Genetic_algorithm}{genetic algorithm} (sometimes called an evolutionary search). In layman's terms, it works like this. We start with 100 very good investment allocation plans (individuals), generated semi-randomly to form our initial population. (We say semi-randomly, because we build these solutions according to common-sense plans we would devise) The algorithm evaluates all the individuals according to our objective of maximising post-retirement income after tax. After evaluating the whole population of individuals, it keeps the best individuals. Each individual has a "genetic code": the string describing the sizes of the contributions and withdrawals to each fund in each year. It uses these codes to mates the best individuals with each other by taking bits of the one and bits of the other to form a new hybrid plan. It evaluates these offspring, and again keeps the best parents and offspring. Random individuals are also mutated semi-randomly (in a way that makes sense) to see if minor changes add any value. If they do, they will persist and become part of the gene pool. By repeating this process for a few generations (20-100), the population gradually improves and produces exceptional individuals, which are kept in the hall of fame. At some stage the population plateaus, and no further improvements are made. At this point the algorithm has converged to a solution. The best individual is then selected as the "winning solution".

\subsection{Assumptions}
\paragraph{Inflation} Because StrataGEM is interested in the optimal investment allocation over time, and not the actual present value of future investments\sidenote{E.g. "to maintain your current living standards, you need to save so much per month"}, inflation doesn't really come into the play. The one exception to this is Capital Gains Tax. Suppose you bought shares for R1m in 2010. If their value grew by the inflation rate, they would be worth roughly $1000000\times1.055^8 = R1.53$m in 2018. In real terms, your assets have not done anything - they have just kept up with inflation. In countries with zero inflation, you might as well have put it under the mattress. But to SARS, you have realised a capital gain of R534 686.50, on which you will pay tax according to the capital gains formula. This sucks. Because of CGT, assets growing at inflation are actually depreciating by the CGT.


\paragraph{Tax rules} \label{tax rules}
If we do include inflation in all our calculations, we should also adjust our tax brackets accordingly. Unfortunately, this isn't realistic. SARS allows "bracket creep" by not adjusting the tax brackets by inflation, thereby collecting more taxes as inflation increases our salaries. But projecting this into the future is difficult. So the best we can do is assume the current tax laws.

Since StrataGEM is concerned with maximising retirement income and not estate planning, the capital gains tax payable on the disposal of assets at the time of death are not considered.

SARS states that 
\begin{quote}
With  the  introduction  of  section 9C, the proceeds  on  disposal  of equity shares (including shares in foreign 
companies listed on a South African exchange) held as trading stock for at least three years are treated as being of a capital nature.\sidenote{\href{http://www.sars.gov.za/AllDocs/OpsDocs/Guides/LAPD-IT-G11\%20-\%20Tax\%20Guide\%20for\%20Share\%20Owners\%20-\%20External\%20Guide.pdf}{Tax Guide for Share Owners, (Issue 5) , p.6.}}
\end{quote}
Some have wondered if one is still liable for CGT if one sells and buys shares every three years to avoid CGT.\sidenote{There is a three-year rule for the disposal of shares, and how they are taxed} However, as Simon Brown points out\sidenote{\href{https://justonelap.com/podcast-cgt-hacking-failed/}{CGT Hacking failed}},
\begin{quote}
The most important factor in determining whether a profit is of a capital or revenue nature is 
the intention of the person when the shares were bought and sold. \sidenote{\href{http://www.sars.gov.za/AllDocs/OpsDocs/Guides/LAPD-IT-G11\%20-\%20Tax\%20Guide\%20for\%20Share\%20Owners\%20-\%20External\%20Guide.pdf}{Tax Guide for Share Owners, (Issue 5) , p.3.}}
\end{quote} Therefore StrataGEM does not bake such strategies in to avoid CGT. It would kinda be the definition of intentionally avoiding tax.

\subsection{What to expect}
The first thing to expect is that you will be taking financial advice from a computer, which feels a bit weird. Of course a lot of human thought and care went into building the program, but you may still be uncomfortable investing money, possibly in counter-intuitive ways, on the advice from a computer. It feels weird to us too. The best advice we have is to check the answers. Try to come up with a better solution yourself by trying different plans. This will give you a feel for the numbers.

The output of the analysis will be a report that shows you how to allocate your funds, and how your capital is expected to grow. You will also get a spreadsheet showing all of the intermediate numbers, in case you are interested.

\newpage
\section{The Gory Details}

\subsection{The Gory Tax Details}

\subsection{The Gory Programming Details}


This document was created in \LaTeX using the Tufte template, available at \url{https://github.com/edwardtufte/tufte-latex}
%\bigskip
%\begin{minipage}{\textwidth}
%\begin{center}
%\begin{tabular}{lcccc}
%\toprule
% & \multicolumn{4}{c}{Books} \\
%\cmidrule(l){2-5} 
%Page content & \vdqi & \ei & \ve & \be \\
%\midrule
%Blank half title page & \hangp{1} & \hangp{1} & \hangp{1} & \hangp{1} \\
%Frontispiece\footnotemark{}
%  & \hangp{2} & \hangp{2} & \hangp{2} & \hangp{2} \\
%Full title page & \hangp{3} & \hangp{3} & \hangp{3} & \hangp{3} \\
%Copyright page & \hangp{4} & \hangp{4} & \hangp{4} & \hangp{4} \\
%Contents & \hangp{5} & \hangp{5} & \hangp{5} & \hangp{5} \\
%Blank & -- & \hangp{6} & \hangp{6} & \hangp{6} \\
%Dedication & \hangp{6} & \hangp{7} & \hangp{7} & 7 \\
%Blank & -- & \hangp{8} & -- & \hangp{8} \\
%Epigraph & -- & -- & \hangp{8} & -- \\
%Introduction & \hangp{7} & \hangp{9} & \hangp{9} & 9 \\
%\bottomrule
%\end{tabular}
%\end{center}
%\end{minipage}
%\vspace{-7\baselineskip}\footnotetext{The contents of this page vary from book to book.  In
 % \vdqi this page is blank; in \ei and \ve this page holds a frontispiece;
%  and in \be this page contains three epigraphs.}
%\vspace{7\baselineskip}

%\begin{figure*}[p]
%\fbox{\includegraphics[width=0.45\linewidth]{graphics/vdqi-title.pdf}}
%\hfill
%\fbox{\includegraphics[width=0.45\linewidth]{graphics/ei-title.pdf}}
%\\\vspace{\baselineskip}
%\fbox{\includegraphics[width=0.45\linewidth]{graphics/ve-title.pdf}}
%\hfill
%\fbox{\includegraphics[width=0.45\linewidth]{graphics/be-title.pdf}}
%\end{figure*}

%\begin{figure*}[p]\index{table of contents}
%\fbox{\includegraphics[width=0.45\linewidth]{graphics/vdqi-contents.pdf}}
%\hfill
%\fbox{\includegraphics[width=0.45\linewidth]{graphics/ei-contents.pdf}}
%\\\vspace{\baselineskip}
%\fbox{\includegraphics[width=0.45\linewidth]{graphics/ve-contents.pdf}}
%\hfill
%\fbox{\includegraphics[width=0.45\linewidth]{graphics/be-contents.pdf}}
%\end{figure*}

%%
% The back matter contains appendices, bibliographies, indices, glossaries, etc.



%\bibliography{sample-handout}
%\bibliographystyle{plainnat}

\end{document}

